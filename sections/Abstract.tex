\begin{abstract}
It is well known that there is a sweet spot on the bat, and it plays
an important role in baseball sport. In this paper, we address the
problem associated with three main portions include the position of
the sweet spot, the influences of the corked bat, and the
differences among different materials bat. We consider two important
models in hopes of solving the problems about the ``sweet spot".

The Plane Mechanics Model established by the \textbf{law of
conservation of momentum} is to solve the three main portions.
First, we analyze the direct impact of the ball and the bat, and
then get the relationship between the speed of the batted ball and
the hitting position, which makes sure the position of the sweet
spot where is not at the end of the bat. This model denies
``corking" a bat enhances the ``sweet spot" effect without
considering the corked bat easily controlled. Eventually, by using
the existing data about the different materials bat we know the
aluminum bat is better than the wood bat.

After presenting the above model, a \textbf{three-dimensional
Computer Model} which can \textbf{randomly simulate} the process of
athletes hit the ball in the ball park. From the simulation result,
We get the range of batted ball and find the standard deviation of
the range using the unmodified bat is larger than using the corded
bat, this is why the corked bat prohibited. On the other hand, it
explains the reason why the aluminum bat is prohibited by Major
League Baseball is that using a aluminum bat is contrary to the
principle of athletic, fairness and safety in the baseball games.

In conclusion, our algorithm is quite easy to implement and to solve
the problem successfully. Though there remain some weaknesses in our
model, it still has the significance for promoting to extensive use.
\end{abstract}
