\section{Results}

Here, comparing the results of each problem obtained in the Computer Model with those that got in the Plane Mechanics, we find that both the two results are surprisingly consistent, which explains that both the previous and the latter model we establish are feasible to some extent. In addition to that, in the Computer Model, we obtain the place of the sweet spot on the bat is not at the end of the bat, too.

Namely, it has also made more convincing evidence to explain whether the corked bat could enhance the ``sweet spot" effect. At the same time, it also better explains the reasons why Major League Baseball prohibit ``corking", which are that the corked bat could extend the reaction time for the batter to hit the baseball, and also let the batter make ball control better. But the level of this kind of the ball control is not able to reflect the athletes�� real competitive level, it is contrary to the principle of the baseball game��s athletic. Besides, the corked bat is easily broken and its safety is not better to use, this may be a reason why it is prohibita (In the USA, the human rights are quite important).

In rapid sequence, we find that the different materials bat have the different influences when hitting the baseball. The effect of the metal bat is better than the wood bat obviously, the most possible reason is that the aluminum bat is hollow when compared with the wood bat with the same mass. This easily makes the baseball obtain a greater speed, which let the competition easier to hit and much more unfair to those who use the wood bats. As a result, for the sake of improving the level of the athletic and fairness about the competition, we think this may be the main reason why Major League Baseball prohibits metal bats.
