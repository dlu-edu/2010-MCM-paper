\begin{thebibliography}{99}

\bibitem{sweetspot}Carello, C., Thuot, S., Andersen, K. L., \& Turvey, M. T. (1999). Perceiving the sweet spot. Perception, 28,1128-1141.
%1
\bibitem{DynamicPerformance}Bryant FO, Burkett LN, Chen SS, KrahenBuhl GS, Lu P. Dynamic and performance characteristics of baseball bats. Rec Q Exerc Sport 1979;48:505-10.
%2
\bibitem{PersonalCommunication}Trey Crisco, Personal communication, Brown University, 1999.
%3
\bibitem{hollowsoftball} Daniel A. Russell, ``The sweet spot of a hollow baseball or softball bat",\\http://paws.kettering.edu/~drussell/bats-new/sweetspot.html,February 19th, 2010.
%4
\bibitem{Percussion} ``Center of Percussion",\\
http://www.fas.harvard.edu/~scidemos/NewtonianMechanics/CenterofPercussion/CenterofPercussion.html,February 19th, 2010.
%5
\bibitem{EvaluatePredict} TIMOTHY J. MUSTONE, A Method to Evaluate and Predict the Performance of Baseball Bats Using Finite Elements. B.S.M.E. UNIVERSITY OF MASSACHUSETTS LOWELL (1996)
%6
\bibitem{SwingWeights} Dan Russell, Kettering University, "Swing Weights of Baseball
and Softball Bats", Submitted to The Physics Teacher, April 17,2009.
%7
\bibitem{Physics} Robert K. Adair, The Physics of Baseball, 3rd Ed., (Harper Collins, 2002)
%8
\bibitem{RemarksCorked} Some Remarks on Corked Bats \\
http://webusers.npl.illinois.edu/~a-nathan/pob/corked-bat-remarks.pdf
%9
\bibitem{Swing_Weights}  Dan Russell, Swing Weights of Baseball and Softball Bats.
%10
\end{thebibliography}
