\section{Strengths and Weaknesses}

\subsection{Model 1}

\textbf{Strengths}

In the Model 1, we take into account the major factors of affecting the ``sweet spot" according to the Law of Conserving Momentum, Law of Conserving Angular Momentum and Coefficient of restitution. By doing this, we are able to obtain the mathematic relationship between the hitting spot and the batting speed, thereby track the so called ``sweet pot". This not only can interpret the ``sweet pot" is not at the end of the bat, but also allows our model to evaluate and compare the ``sweet spot" effect of different bats which based on different parameters.

The results not only are consistent with the conclusions of related references, but also accord with the computing model. Using this model, our approach allowed us to isolate the random factors and the three-dimensional space, so as to obtain the certain results according to the specific parameter.


\textbf{Weaknesses}

While our model attempted to model the ``sweet pot" effect of different bat based on different parameters accurately, it was of cause impossible to completely capture random factors, including the velocity of ball and bat before hitting and hitting spot. Further, we just only take into account the one-dimensional space (direct impact), which lead to inevitable restriction.

In addition, we only considered value of the velocity before hitting, while we did take into account the direction of the velocity, so that we can not calculate the shot distance of the ball. While using the computing model did allow us to improve upon the model 1, further improvements will be available by examining other possibilities.

\subsection{Model 2}

\textbf{Strengths}

In the model 2, we modify the model from one-dimensional to three-dimensional based on the model 1.That means we not only consider the direct impact, but also take into account skew impact, which is more consistent with actual hitting process. Further, we consider the random factors, including the velocity of ball and bat before hitting and hitting spot, which accord with the actual case.

In addition, by doing this, we not only can simulate the hitting process of different baseball player and different bat, but also we can calculate the shot distance and its distribution function. This model has solved the restrictions of model 1, which make further improvements available.

\textbf{Weaknesses}

While this model can consider the random factors of hitting process, we have ignore the friction between the ball and the bat, so as to consider that the ball cannot circumgyrate, which is not consistent with the actual process. However, the friction is so small compared with hitting impact and the hitting time is so extremely short that the friction has little impact on the results. Hence, the error is neglect able.

In addition, we ignore the effect of the air friction on the shot distance, which makes the results of shot distance bigger than the actual value. While in fact this will not affect the comparisons of ``sweet spot" effect between different bats based on different parameters.


