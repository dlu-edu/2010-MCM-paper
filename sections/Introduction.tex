\section{Introduction}

With the popularity of the baseball, more and more people like the baseball sport. It is well-known to every hitter that there is a point which is called ``sweet spot" on the fat part of a baseball bat where maximum power is transferred to the ball when hit. And some players believe that ``corking" a bat (hollowing out a cylinder in the head of the bat and filling it with cork or rubber, then replacing a wood cap) enhances the ``sweet spot" effect. And more than that, the different materials bat will also affect the distribution of the sweet spot on the bat.

\subsection{Problem  Background}

On June 3, 2003, Chicago Cubs centerfielder Sammy Sosa was ejected from a game in the first inning for using a corked bat. His bat shattered upon impact with the ball and the umpire who picked it up discovered the bat had been hollowed out and filled with cork. Sammy said it was an accident that he had used the bat during batting practice and accidentally grabbed it by mistake when he went to the plate. Regardless of how the bat got there, it was not the first time a player has been caught with a doctored bat, and it won't be the last. But why not one raises the question ``Does the corked bat really enhance the effect of the `sweet spot'?" Is there any advantage (backed up by science) to using a corked bat?

Perceiving the sweet spot have conducted experiments directed at the question of whether one can perceive the sweet spot of a striking implement simply on the basis of wielding the implement and not just at the time of contact\cite{sweetspot}.

Trey Crisco et. al.\cite{DynamicPerformance} give a simulation only requires replicating the bat motion during the instant of contact with the ball, however invovles pure rotation.
Over the years, papers in this journal have addressed both experimental and theoretical issues associated with the baseball-bat collision. Bynamic FO et. al.\cite{PersonalCommunication} proves that the moden alumunim ans composite bats may hit a ball an average of 1 m/s(4 mph) faster than a traditional wooden bat.

\subsection{The explanation of the sweet spot}

Trying to locate the exact sweet spot on a baseball bat is not as simple a task as it might seem, because there are a multitude of definitions of the sweet spot. At present, there are seven different definitions about the sweet spot as followed:\cite{hollowsoftball}

\begin{itemize}
\item The location which produces least vibrational sensation (sting) in the batter's hands.

\item The location which produces maximum batted ball speed.

\item The location where maximum energy is transferred to the ball.

\item The location where coefficient of restitution is maximum.

\item The center of percussion.

\item The node of the fundamental vibrational mode.

\item The region between nodes of the first two vibrational modes.

\item The region between center of percussion and node of first vibrational mode.

\end{itemize}


From a player's viewpoint, the sweet spot is the place on the bat barrel where the contact between the bat and the baseball results in the best hit, the ball with the greatest speed when leaves the bat and his hands feel very little even no vibration from the impact. By looking up in lots of reference sources, we find there are two most frequent points of view about ``sweet spot", one regards it as the center of percussion (COP), which is the place on a bat or racket where it may be struck without a causing reaction at the point of support \cite{Percussion}, and the other is that it is the spot where the baseball gets the maximum instantaneous velocity when shot out of \cite{hollowsoftball}. In order to make it clear that what is the ``sweet spot", the first aspect that we should not only take into major consideration the speed of the ball gets, but also obtain some important findings through researching to make it apparent to the comfortable standard of the player.


